\section{Conclusions and Future Work} 
In this paper, we have initiated a model which decentralizes the location of 
shared content within a P2P structure for a CSCW application (VoIP). 
As a part of future work, we aim to extend this application for people who can 
download the voice stream from the presentation remotely in real-time. We also 
aim to use a hierarchical structure for the network which will help the 
different nodes present in the network to change their roles during the 
presentation at any time. Once the application is in operation for a limited 
number of users, we aim to extend this application in a way so that it 
can be used for any number of users without having any storage issues, any 
disruption in the voice streaming or any bandwidth issues in the network by 
adding large number of users to the presentation. Another thing that we hope to 
address in the future is that if any of the relayer nodes fails then all 
the leaf nodes linked to that relayer will be dynamically transferred to 
another relayer. New schemes will be developed to guarantee that these dynamic 
changes will not significantly reduce the network performance, as well as to 
keep a balanced traffic flow with the optimal bandwidth utilization.