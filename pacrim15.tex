%\documentclass[a4paper,10pt]{report}
\documentclass[conference]{IEEEtran}
\usepackage[utf8]{inputenc}

% Usepackages, could be moved to a new file
% Long Table and decimal aligned columns
\usepackage{dcolumn}
\usepackage{longtable}

% Mathematics support
\usepackage{amsmath}
\usepackage{amsthm}
\usepackage{amssymb}


% Text Control
\usepackage{xspace}
\usepackage{textcase}
\usepackage{url}%hyperref}

% Graphics
\usepackage{wasysym}
\usepackage{graphics}
\usepackage{graphicx}   % A package to allow insertion of
                        % external image files
\usepackage{epstopdf}
% Comments
\usepackage{comment}
% Citation
%\usepackage{natbib}


\newif\ifrev
% comment out the following line after the revision is proved
%\revtrue
\ifrev
  \usepackage{color}
  \usepackage[normalem]{ulem}
  \newcommand{\add}[1]{\textcolor{blue}{\uline{#1}}}
  \newcommand{\del}[1]{\textcolor{red}{\sout{#1}}}
\else
  \newcommand{\add}[1]{#1}
  \newcommand{\del}[1]{}
\fi

% Title Page
% \title{Design and Implementation for a Peer-to-peer-based Mobile VoIP 
% Application}
\title{ThinkTogether: A Collaborative App based on a Peer-to-Peer 
Infrastructure}
\author{\qquad Tianming Wei \qquad Yongjun Xu \qquad Yiyun Zhao \qquad Nishant Khanna \qquad Bing Gao \qquad Yvonne Coady\\
Department of Computer Science\\
University of Victoria, Victoria, BC, Canada\\}


\begin{document}
\maketitle
\pagestyle{plain}

\begin{abstract}
\begin{comment}
With the enhancement in technology every day there are so many applications 
being developed every day. There are a lot of applications that have been 
developed to record presentations and to upload content onto the server for 
later reference. ThinkTogether is much more than that. Not only does it support 
basic functions like giving presentation, and adding comments and marking 
important points, ThinkTogether also provides multilayer function, which achieve 
better performance by separating different users into multiple layers. As fully 
functionable as the app already is, we still want to take it to the next level. 
In this study, we propose a network mechanism using VoIP technology, to 
introduce voice streaming feature into existing application developed by Two 
Tall Totems.
\end{comment}
Computer Supported Collaborative Work (CSCW) has been greatly enhanced by 
technology that provides almost real-time capture, replay, and sharing of 
content.   “The cloud” has popularized ease of deployment for apps serviced 
through a centralized approach.  In this work we propose a more aggressive 
design, decentralizing the location of shared content within a peer-to-peer 
structure.  Ease of deployment is maintained through a novel architecture for a 
multilayer distributed cloud.  This paper describes how this approach has shown 
promise in a prototype app, {\em ThinkTogether}, and the potential to enable a 
consistent user experience, even for remote peers sharing VoIP.

\end{abstract}

\begin{IEEEkeywords}
  Voice over IP, Peer-to-peer, Quality of service, Traffic ballance, Bandwidth utilization.
\end{IEEEkeywords}

%\section{Introduction}
\section{Introduction}

ThinkTogether is a mobile collaboration platform developed by Two Tall Totems, which is also a creative Bring Your Own Device (BYOD) solution. It enables teachers and students to cooperate together through the lecture in a class or presenters and attendees to participate together in a conference. The current implementation of the project on iOS supports the most functionality of the presentation, yet not with voice stream. It is essential to support the voice stream in the application for remote education since students will be able to have a better interaction with the class even if they are not in the classroom. We are going to design and implement a Peer-to-Peer (P2P) based Voice over IP (VoIP) system on the top of the existing application on iOS so that the users not only could actively enjoy the presentation on live but also could store and replay it later on the cloud.

Due to a variety of factors that could affect the quality of service (QoS) for the application, such as limited bandwidth, high volume traffic, network delay and effect of peer signal, we are going to explore and analyze a better solution with optimized performance. This project aims to create a new education model and subvert the traditional way of learning, sharing and communicating with mobile devices through different cutting-edge mobile technology. First of all, we will model a hierarchical architecture, and verify the model as well as the design with a simulation software Omnet++. After that, analysis will be applied to the data collected from the results of the simulation and we are going to discover the bottleneck in our VoIP solution as well as make an improvement.

Therefore, we propose to experiment the method of pushing data among all the devices and ensure the usage of bandwidth that is enough for each user as well as for the cloud server. We are going to implement a stable service based on Voice over IP (VoIP), Peer-to-peer (P2P) or Multi-peer technology, which would also be scalable for large amount users with mobile devices in the existing App. With our partner Two Tall Totems, development on this project with these advanced communication and network technology could greatly change how people think together and help to build a better world.


%\section{Related Work}\label{simu}
\section{Related Work}

\begin{comment}
Voice over IP (VoIP):

There are various studies that have been done in the field of VoIP (Voice 
over IP) which is a technology used 
for voice communication over multimedia devices with the help of the IP 
protocols. Some of the related work has been done by Wook \& Kang [1] where 
their main focus is based on VoIP using SIP( Session Initiation Protocol) for 
smartphones working on iOS, Android \& Windows based OS , In our project we’ll 
be using a similar technology of VoIP but our focus is on iOS devices i.e. 
mostly IPAD’s. Similarly there is some other research done by Fathi et al. [2] 
for the optimization of VoIP services in Wireless networks and how to reduce the 
disruptions in real-time sessions by focusing on the network layer and not so 
much on the link layer, In our project we aim to use a similar concept where we 
aim to reduce delay in voice streaming for live presentations for Wireless 
devices with a focus on both the network and link layer. Work done by Rao et al. 
[3] is based on providing VoIP services for GSM based devices without making any 
modifications in the GSM network, we use a similar concept for providing VoIP 
services but with a focus on iOS devices instead of GSM devices . 
There is some work done to improve the VoIP capacity in Wireless networks done 
by Fong et al. [4], we plan to use a similar concept as presented in the paper 
[4] to handle voice traffic as with the voice streams during a live presentation 
there is a chance that the voice traffic might increase and it is very important 
to handle this traffic in the best possible way to avoid any disruption in voice 
streaming in real-time. All these studies talk about VoIP applications and how 
to improve the capacity of such systems in Wireless networks.
Now there some research done for VoIP in mobile devices like: Aggregation of 
VoIP streams done by Komolafe \& Gardner [5], we focus on using a similar 
concept but instead of aggregation of calls we are concerned with aggregation of 
voice streams.
The work done by Agarwal et al. [6] is based on energy management of VoIP over 
Wi-Fi for Smartphones, In our project instead of focusing on Smartphones we 
focus on iOS devices (IPAD’s) , but we also use a similar concept for energy 
management of VoIP over the devices and with increasing amount voice streaming 
over the network the energy of the device needs to be managed in an efficient 
and environment friendly manner.  Quality Measurement of VoIP over mobile 
devices by Chen et al. [7] gives some methods to measure the Quality of VoIP 
over mobile devices, we plan to use a similar concept as voice quality is one 
the most important aspects in our project, so we need to ensure that voice 
quality is perfect without any kind of noise or disruption in voice streaming 
over the network. Performance Measurement of VoIP over the mobile networks by 
Kim et al. [8] provide some methods to measure the performance of VoIP, we will 
be using a similar approach as proposed in the paper [8] to measure the 
performance of VoIP over the mobile networks and by using the results of 
performance measurement we would analyse the network and try to improve any 
possible disruption that might cause any kind of delay in streaming voices over 
the network.
There are some other models where VoIP has been implemented using P2P 
(Peer-to-peer) like: Improving performance of P2P based VoIP applications by 
Wang et al. [11] where they analysed the delay in the Chinese internet system. 
To analyse the delay different methods were used in the paper [11], we aim to 
use a similar kind of technique to analyse the delay of voice streaming in our 
application.


Skype:
There are some existing technologies that use the VoIP feature like Skype. Lot 
of research has been done in this field like, An experimental study on Skype 
P2P VoIP Systems done by Guha \& Daswani [12] where they study different 
characteristics of network traffic from data derived from online users in the 
world, Based on the results obtained in this study we aim to analyse those 
results and will try to employ a similar approach used in the P2P VoIP 
technology used in Skype to our own project. Work  done by Hoßfeld \& 
Binzenhöfer [13] where they analyze the Quality of service (QoS) of the 
end-to-end connection and the Quality of Experience (QoE) of the end user, we 
aim to use a similar kind of approach to analyze the QoS \& the QoE of the 
application ThinkTogether so that if there is any problem in any of these 
measurements it can be fixed in the best possible way so that we can achieve our 
goal of implementing voice streaming in the application with the least amount of 
disturbance while transferring voice over the network. Analysis of the Skype P2P 
Internet telephony protocol by Baset and Schulzrinne [14] aims at studying the 
voice quality that has been implemented in Skype and how it is better than other 
applications that have also implemented a similar feature, In our project we aim 
to use a similar concept that has been implemented in Skype but our focus is 
only on the voice part and not the video part that has been implemented in 
Skype.
In our project we'll be using all these technologies and studying some other 
works that has been  done on them and use those ideas to implement VoIP function 
in ThinkTogether.


Peer-to-peer(P2P) \& Client/Server model(C/S):

There are two widely adopted networking model, Peer-to-peer(P2P) and 
Client-server(CS). Peer-to-peer is a networking mechanism in which each machine 
acts as server and clients at the same time. The roles are interchangeable 
simultaneously, dividing the tasks into a set of individual machines. In 
contrast to Server-Client model in which the supply of resource is separated 
from the consumption of resource, there is no explicit division in terms of work 
allocation in peer-to-peer networking. This eliminates the need to establish 
central hosts or servers for coordination. 

The concept of Peer-to-peer model used in transferring voice stream from one 
node to another has been discussed throughout years. Several connection models 
using P2P protocol have already been proposed to facilitate the communication 
between multiple handheld devices. A paper[9] done by S.Sundar et al put forward 
a novel algorithm along with suggested architecture to support voice 
transmission among mobile phones. Short-distance mobile communication is set up 
through Bluetooth, which is an integrated facility for most mobile phones 
nowadays; while long-distance calls are connected through WIFI. The mechanism 
developed in the paper proceed with the calls using GSM technology if no 
cost-free networking is accessible. Our primary focus is managing data while 
connected to network; yet the knowledge using GSM provides insight of future 
development. Ghassan Kbar et al[10] also dive in  research of similar topic. 
These two papers concentrate on eliminating costly GSM(Global system for Mobile) 
telecommunication by exploring other channels such as Bluetooth and WIFI. The 
work presented in papers relates voice implementation in ThinkTogether app, 
typically in the first stage - connection. ThinkTogether project absorbs the 
idea of how to design communication model to be adaptive of current network 
situation. 

To compare with Peer-to-peer, another networking mechanism:server-client model 
is also introduced in the project. Each model will be evaluated in terms of 
performance in voice data transmission. The one with better performance in voice 
stream will be chosen. The paper proposed by Fobert et al[15] provides a general 
look into IP telephony in Server/Client Model. It features components included 
in managing voice data packets. The paper was published in early years when IP 
telephony was immature, therefore only providing a fundamental principle of 
transferring voice data over internet in C/S model. However, simulation will 
built on that knowledge to determine which candidate model, either C/S or P2P, 
benefits ThinkTogether the most.

\end{comment}


Our case study focuses specifically on VoIP (Voice over IP), which is a 
technology used for voice communication over multimedia devices with the help of 
the IP protocols. Wook \& Kang \cite{HS_VoIP_ICTC10} focused on VoIP using SIP 
(Session 
Initiation 
Protocol) for smartphones in general, whereas we hope to extend some of these 
results to iPads.  Fathi et al.~\cite{HSR_VoIP_TVT07} optimized VoIP services 
in wireless networks 
and reduced the disruptions in real-time by focusing on the network layer as 
opposed to the link layer.   In our work we aim to use a similar concept, where 
we aim to reduce delay in voice streaming for live presentations for wireless 
devices with a focus on both the network and link layer.  
 
There are many studies on VoIP applications and how to improve the capacity of 
such systems in Wireless networks.  For example, Rao et 
al.~\cite{HYS_VoIP_COM00} provided VoIP 
services for GSM (Global System for Mobile) based devices without making any 
modifications in the GSM network.  Similar work by Fong et 
al.~\cite{MRSR_VoIP_COM08} focused on 
Windows.  Approaches like these are great examples of how to handle traffic 
dynamically, to avoid any disruption in voice streaming in real-time. Komolafe 
\& Gardner~\cite{OR_VoIP_PMC03} studied aggregation of VoIP calls, whereas we 
focus on 
aggregation 
of voice streams. Agarwal et al. \cite{YRAP_VoIP_MSYS07} consider energy 
management of VoIP over 
Wi-Fi for Smartphones, which should be directly applicable to our case study 
using iPads.  An increase in voice streaming will most definitely require the 
energy of the device to be managed in an efficient and environmentally friendly 
manner.   Chen et al. \cite{WPY_VoIP_SYS11} provide methodology to measure the 
Quality of VoIP 
over mobile devices, and Kim et al. \cite{DHMS_WiMAX_WOW08} provide metrics for 
performance of VoIP 
over the mobile networks, and we intend to leverage these same methodologies in 
our work.  Related work studying VoIP using P2P design includes Wang et al. 
\cite{GCXZ_P2PVoIP_MTA13}, where they analyzed the delay in the Chinese internet 
system, which also 
factors in as an important concern in our case study.
 
An experimental study on Skype P2P VoIP Systems by Guha \& Daswani 
\cite{SN_Skype_CIST05} showed 
the variety of characteristics of network traffic from data derived from online 
users in the world.  We hope to leverage these insights in our work, along with 
work by  Hoßfeld \& Binzenh ofer \cite{TA_Skype_COMPNET08} focused on Quality of 
service (QoS) of 
the 
endto-end connection and the Quality of Experience (QoE) of the end user. A 
similar analysis of the Skype P2P Internet telephony protocol by Baset and 
Schulzrinne \cite{SH_Skype_INFO06} establishes the superior voice quality in 
Skype relative to 
other state-of-the-art applications.
 
Sundar et al \cite{SMPM_VoIP_ICAESM12} put forward a novel algorithm along with 
suggested architecture 
to support P2P voice transmission among mobile phones.  Short-distance mobile 
communication is set up through Bluetooth, while long-distance calls are 
connected through WIFI. The mechanism developed uses GSM technology if no 
costfree networking is accessible. Our primary focus is managing data while 
connected to network; yet GSM provides insight of future development.  Ghassan 
Kbar et al \cite{GWA_P2P_VoIP_ICWMC10} also concentrate on eliminating costly 
GSM telecommunication by 
exploring other channels such as Bluetooth and WIFI.  Our goal is to extend the 
voice feature of the ThinkTogether app to be adaptive of current network 
situation.
 
Fobert et al \cite{JSPS_USPatent05} provides a general look into IP telephony 
in Server/Client 
Model managing voice data packets.  The paper was published in early years when 
IP telephony was immature, therefore only providing a fundamental principle of 
transferring voice data over internet  We use this as a comparison to the 
proposed P2P architecture in terms of performance in voice data transmission 
within the ThinkTogether app.






%\section{System Model}
\section{Network Scenario and Modeling}

\begin{figure}[h!]  
  \centering  
    \includegraphics[width=0.5\textwidth]{figures/model1_test2.jpg}
  \caption{P2P-based Model for VoIP}
  %\vspace{-1in}
  \label{fig:mod1}  
\end{figure}
There are four main kinds of components/nodes in the designed scenario, a 
presenter, multiple relayers and listeners, and a remote cloud server. Each 
component can coordinate with one another in order pass streaming content such 
as the voices or camera videos of the speaker, to the listeners in real-time. As 
showed in Fig.~\ref{fig:mod1}, the network is organized in a hierarchical 
architecture. The presenter will be producing the streaming content from the 
root node and broadcast to the relayers and listeners, as well as to 
the cloud servers for backup. 

For example, one can think of the model as a lecture in real life. A presenter 
is an instructor who gives a lecture at the front of the room, and the 
listeners are 
students who are listening to the lecture (can also be remote listening). When 
a 
presenter creates a presentation, the voices or videos being recorded will be 
sent to both the child nodes and the remote cloud server so that the servers can 
maintain a full copy for the streaming content in the presentation. All the 
other nodes can passively receive the packets from the presenter.
%The relayers and cloud server are possible approaches the instructor adopts to 
%pass on that voice stream. 
%How well a student understand is not a good 
%predictor of instructor’s capability; it largely depends on the media, that 
%is, how well the instructor could transfer his thinking in a explanatory way. 
%And that is where the cloud and relayers come into play.
There are different routes a voice packet can travel through. The 
listeners receive voice packets through the relayers while the cloud server 
can provide back-up services. If any of the listeners or relayers encounter a 
data loss due to a bad network condition, the node can send requests to the 
cloud server and retrieve the missing content. Once a relayer goes down, 
the listeners will request the voice stream from cloud. The cloud will take 
over 
and serve as the main approach to deliver packets. 
%Noted that only in simple cases where all attendees share the same role do 
%relayers act as routers. 
For a more complex scenario, where there are multiple roles for the listeners, 
relayers act as filter and only forward packets to the designated attendees. 

\begin{figure}[h!]  
  \centering
      \includegraphics[width=0.5\textwidth]{figures/model2_test2.jpg}
  \caption{Relayers Become Presenters}
  \label{fig:mod2}
\end{figure}

However, this simple hierarchical architecture cannot be adaptable to 
dynamic network changes and could experience an unbalanced network 
traffic problem which could seriously reduce the quality of serve and 
experience. For example, Fig.~\ref{fig:mod2} shows that if a 
high-level (near to the root) relayer suddenly leaves the network, all the 
nodes lower than the relayer will not be able to receive the streaming content 
any more, and thus the server will receive a large quality of requests from the 
nodes which loss the connection from the presenter. This will cause the server 
bandwidth to be occupied by these nodes which will make the network flow 
unbalanced. Also, the relayers among these nodes could lose their advantage in 
passing the content to the listeners, as the listeners could already obtain the 
requested packets from the cloud server and bandwidth between the relayers 
and the listeners is wasted. 

Therefore, our proposed solution is to design a P2P-based 
architecture among the relayers 
and listeners. The server could act as the tracker in this scenario. Every node 
will have to register with the server first when it joins in a presentation so 
that the server will be aware of the network situation for each node. According 
to the information the server received from each node (including bandwith, 
account information based on the future business model, etc.), some nodes will 
be selected as relayers at different levels, while the other nodes will act as 
the listeners (leaves) and establish connections with the relayers. The nodes 
at the same level will be connected as well, to establish the P2P-based 
architecture, as shown in Fig.~\ref{fig:mod3}.


\begin{figure}[h!]  
  \centering
      \includegraphics[width=0.5\textwidth]{figures/model3_test2.jpg}
  \caption{Relayers Become Presenters}
  \label{fig:mod3}
\end{figure}


\begin{comment}
Besides, the concept of background channel will be applied for transmitting the 
data in the background. Apart from the live channel which ensures the client to 
have the presentation on live, there will also be a background channel to help 
to upload or download previous missing data that make the data of the 
presentation synchronized and completed in both server and client side while the 
clients won’t even notice.
\end{comment}









%\section{The heuristic algorithm}\label{alg}
\section{Analysis and Case Study}
We used Omnet++ as a simulation tool to explore the most 
suitable model and algorithm for ThinkTogether. The study done by 
Varga~\cite{A_OMNET_ESM01} 
provides brief introduction of Omnet++, including Model Libraries, NED 
language, 
model structure. The paper also describes how to execute the simulation under a 
powerful graphical user interface, which will help the team to better utilize 
the tool. Work done by Anggadjaja et al.~\cite{EI_OMNET_ACT10}. features the 
use of Omnet++ based 
simulation for reliable point-to-point wireless transmission. The outcome could 
provide inspiration for employing P2P model in wireless network. There 
are several other papers~\cite{MM_OMNET_OMN08,XX_OMNET_ICQ12} explaining the 
main infrastructure and 
internal 
design of Omnet++. 
%The team can gain overall understanding of the tool, through 
%extensive papers above, and use Omnet++ with efficiency and accuracy. 

%Experimental data is the most scientific and objective way of verifying the 
efficiency of a particular network. There comes the simulation. 
In our work, 
two network models will be built in order to observe the behavior of data 
packets transferred in between. One of networking protocol is Telnet, on which 
J. Postel et al~\cite{JJ_TEL83} provides specification and 
standard. This is 
a relatively old paper that is published in early 1980s. The paper draws a 
whole picture of telnet protocol from general consideration to signal 
processing. The team refers to this paper to get a understanding of how Telnet 
protocol is put to use. Another protocol adopted is HTTP. There are wider range 
of papers dedicated to HTTP. One done by R Fielding~\cite{BJJH_HTTP1999} 
provides a full-scaled overview, which is used as a reference for modelling. 

%Multilayer Transmission

ThinkTogether is a communication application that contains multiple layers in 
packet transmission. Data sent by each individual is handled differently. G. 
Sekin et al~\cite{GF_COM00} provides an insight of how to manage multi 
priority traffic in ATM network. The study is content-based video 
objects. As the common characteristics shared between video stream and voice 
stream, such as low tolerance in delay, jitter and packet misorder, our team 
could adopt ideas proposed in the paper. Unfortunately, there is no evaluation 
involved in the paper, therefore the feasibility of proposed algorithm remains 
unknown. 


After designing the model, we plan to create a working 
network by creating a simulation using Omnet++, as shown in Fig.~\ref{fig:sce}. 
By creating 
this simulation we can have an idea of how the voice stream packets are being 
transferred between the different nodes present in the network. Once the 
simulation starts working for a basic scenario of just 2 leaf nodes, a relayer 
\& a cloud server, we will add more number of nodes to the simulation and then 
analyse what changes are to be made to certain parameters of the network so as 
to avoid any kind of latency and disturbance while transferring the voice 
streams. 

Presenters initialize services by sending voice packets to both relayers and 
cloud. Upon receiving the packets, the server stores them 
locally and echoes back an acknowledgement. The services between presenter and 
server is now finished. In other side,
voice packets reach the relayer and will be directed to all attendees relayer 
connects to, which is illustrated by \#6 and \#7 event in Fig.~\ref{fig:r_seq}. 
Out of 
testing purpose, each attendees request the exact same voice data from cloud. 
Cloud handles that request by fetching corresponding data in server and sending 
back the requested content. One can refer to events from \#11 to \#16 for 
execution sequence details. 
 

 
\section{Implementation: Data Distribution and Layers}

The simulation tests the efficiency in transferring voice data packets 
with a fundamental model. This model lays the groundwork for future expansion, 
and illustrates a simple way to consider a classroom scenario. The purpose of 
simulation is 
to investigate the feasibility, and more importantly, the efficiency of the 
model. In the simulation, parameters such as packet drops, delay time, etc. are 
collected and extracted to generate an output file for further analysis. The 
output file is then used as 
a foundation for performance analysis.  


%In this paper our aim is to deploy a distributed the feature of voice 
%streaming in the 
%already developed application of Two Tall Totems called Think Together. In 
%order to implement this feature we have 
In our simulation of the model architecture, we use multiple nodes 
as the leaf nodes in a network. 
Then we also have some nodes that act as relayers in the network that will 
transfer the voice stream from the presenter in the network to 
the corresponding leaf nodes in the network. Along with all these nodes we also 
have a cloud server which will act as a primary storage device for all the 
voice streams that will transmitted throughout the network. All the nodes will 
have a direct link to the server so that if any of the leaf nodes (student 
nodes in our scenario) wants to retrieve any voice stream they can do so by 
making a request to 
the cloud server for that voice stream directly through the relayer (big node). 

To simplify, we assume all the attendees are physically located in the same 
classroom and they are subscribing to voice for the purposes of capturing the 
information. This will eliminate potential influence caused by the difference 
of 
location, such as propagation delay. There are five components in the model: 
presenter, relayer, cloud, server and attendee. Each component coordinates with 
one another in order to offer recording services to attendees. There are two 
test cases, one as a small-scaled network with two attendees, the other as 
slightly larger-scaled network with ten attendees. The simulation 
is run in HTTP net, 
masking the lower level network layers such as TCP and IP. 


%Model Taxonomy

Figure~\ref{fig:sce} shows the overall picture of the simulation model in 
small-scaled case. It 
breaks down to four  parts: sender (presenter), receiver (attendees), 
mediator (relayers), and backup (cloud and server). 

\begin{figure}[h!]  
  \centering
    \includegraphics[width=0.5\textwidth]{figures/sce.png}
  \caption{Simulation in Omnet++}
  \label{fig:sce}
\end{figure}

Presenter is the starting point of service. It does not rely on any outer 
resource to trigger sending process. Currently no packets other than 
acknowledgement are sent to presenters. In other words, one can view presenters 
as a resource provider and there is no prerequisite for presenters to function. 
There are two outgoing links connected to presenters. The one connected to 
relayers is the main channel for delivering data. The one connected to cloud 
transfers the same content, yet it is less demanding in terms of propagation 
speed and processing time. A presenter sends voice packets to both links 
simultaneously, and sets a timer for next cycle. 

The cloud server can be considered as a group in terms of services 
purpose. 
They 
are both built to provide backup for attendees. There are two scenarios in 
which 
attendees would need backup services. The first is when packets are lost by the 
relayers and attendees will ask for complementary data from cloud to ensure a 
smooth stream. The second is when attendees are not able to attend 
the 
class, but request a later review of that class. The server stores the 
lectures in the format of voice stream and delivers it to attendees via cloud. 
In this regard, the cloud is an interface between service providers and 
receivers. In addition, once network is scaled up to include multiple servers, 
cloud acts as coordinator that organizes servers’ actions. 

The relayer acts as a router in this simple case simulation. Unlike servers, 
the relayer 
does 
not store voice data, thereforegull attendees can not request data from relayer 
later on. It should be noted that this primary function is not yet implemented, 
and will be 
addressed later in future model. 

Attendees are students in the diagram. They receive voice packets from 
presenters, regularly through relayers. It should be noted that attendees will 
not send 
acknowledgement back to presenters. When the network between relayers and 
attendees are down, or for some reason, packets are lost in the regular 
transmission routes, attendees will request the data from the server. The 
target 
users of ThinkTogether app is attendees and presenters.  
%Execution Sequence:

To collect data from multiple cycles, we set a timer for both presenter and 
attendees based on an exponential distribution. After sending the voice 
packets, 
the presenter will create a timer that controls how long before the next cycle, 
namely, the next time it starts sending the packets. The duration of the timer 
is based on a exponential function, which allows random distribution in sending 
packets. The same principle applies to rate of sending request from attendees. 
There is also a timer created by attendee that controls the rate of sending 
requests to cloud. In real life, an attendee need complementary data from cloud 
only when relayers fail to deliver it. However, in order to fully evaluate the 
function, in the simulation we configure an attendee to request voice data from 
the cloud whether the relayers go down or not.

\begin{figure}[h!]
  \centering
    \includegraphics[width=0.5\textwidth]{figures/r_seq.png}
  \caption{Events in Time Sequence}
  \label{fig:r_seq}
\end{figure}

The execution sequence is illustrated by diagram below. It is a log file 
generated by running a small-scaled cases. 

%\section{Deployment: Distributed Cloud}




%\section{Performance Evaluation}\label{simu}
\input{sec_eval}

%\section{Conclusion and Further Work}\label{conclu}
\section{Conclusions and Future Work} 
In this paper, we have initiated a model which decentralizes the location of 
shared content within a P2P structure for a CSCW application (VoIP). 
As a part of future work, we aim to extend this application for people who can 
download the voice stream from the presentation remotely in real-time. We also 
aim to use a hierarchical structure for the network which will help the 
different nodes present in the network to change their roles during the 
presentation at any time. Once the application is in operation for a limited 
number of users, we aim to extend this application in a way so that it 
can be used for any number of users without having any storage issues, any 
disruption in the voice streaming or any bandwidth issues in the network by 
adding large number of users to the presentation. Another thing that we hope to 
address in the future is that if any of the relayer nodes fails then all 
the leaf nodes linked to that relayer will be dynamically transferred to 
another relayer. New schemes will be developed to guarantee that these dynamic 
changes will not significantly reduce the network performance, as well as to 
keep a balanced traffic flow with the optimal bandwidth utilization.

%\input{appendix}


\bibliographystyle{plain}
\begin{thebibliography}{10}

\bibitem{IEEEhowto:kopka}
H.~Kopka and P.~W. Daly, \emph{A Guide to \LaTeX}, 3rd~ed.\hskip 1em plus
  0.5em minus 0.4em\relax Harlow, England: Addison-Wesley, 1999.
  
\bibitem{sample}
\newblock ``Efficient user-assisted content distribution over 
information-centric
  network,''
\newblock In {\em Proc. IFIP Networking Conference}, pp. 1--12, 2012.

%ref 01
\bibitem{}
Hyun Wook; ShinGak Kang, "Design and implementation of SIP-based mobile VoIP 
application for multiple smartphone OS," Information and Communication 
Technology Convergence (ICTC), 2010 International Conference on, pp.565,568, 
17-19 Nov. 2010. doi: 10.1109/ICTC.2010.5674765

%ref 02
\bibitem{}
Fathi, H.; Chakraborty, S.S.; Prasad, R., "Optimization of Mobile IPv6-Based 
Handovers to Support VoIP Services in Wireless Heterogeneous Networks," 
Vehicular Technology, IEEE Transactions on , vol.56, no.1, pp.260,270, Jan. 
2007. doi: 10.1109/TVT.2006.883806

%ref 03
\bibitem{}
Rao, H.C.-H.; Yi-Bing Lin; Sheng-Lin Cho, "iGSM: VoIP service for mobile 
networks," Communications Magazine, IEEE , vol.38, no.4, pp.62,69, Apr 2000. 
doi: 10.1109/35.833558

%ref 04
\bibitem{}
Mo-Han Fong; Novak, R.; Mcbeath, S.; Srinivasan, R., "Improved VoIP capacity 
in mobile WiMAX systems using persistent resource allocation," Communications 
Magazine, IEEE , vol.46, no.10, pp.50,57, October 2008 doi: 
10.1109/MCOM.2008.4644119

%ref 05
\bibitem{}
Komolafe, O.; Gardner, R., "Aggregation of VoIP streams in a 3G mobile 
network: a teletraffic perspective," Personal Mobile Communications Conference, 
2003. 5th European (Conf. Publ. No. 492) , vol., no., pp.545,549, 22-25 April 
2003. doi: 10.1049/cp:20030314

%ref 06
\bibitem{} 
Yuvraj Agarwal, Ranveer Chandra, Alec Wolman, Paramvir Bahl, Kevin Chin, and 
Rajesh Gupta, “Wireless wakeups revisited: energy management for voip over wi-fi 
smartphones,” In Proceedings of the 5th international conference on Mobile 
systems, applications and services(MobiSys '07). ACM, New York, NY, USA, 
179-191. doi=10.1145/1247660.1247682

%ref 07
\bibitem{}
Whai-En Chen; Pin-Jen Lin; Yi-Bing Lin, "Real-Time VoIP Quality Measurement 
for Mobile
Devices," Systems Journal, IEEE , vol.5, no.4, pp.538,544, Dec. 2011.
doi: 10.1109/JSYST.2011.2165604

%ref 08
\bibitem{}
[8] Dongmyoung Kim; Hua Cai; Minsoo Na; Sunghyun Choi, "Performance measurement 
over Mobile WiMAX/IEEE 802.16e network," World of Wireless, Mobile and 
Multimedia Networks, 2008. WoWMoM 2008. 2008 International Symposium on a , 
vol., no., pp.1,8, 23-26 June 2008
doi: 10.1109/WOWMOM.2008.4594817
 
%ref 09
\bibitem{}
Sundar, S.; Kumar, M.K.; Selvinpremkumar, P.; Chinnadurai, M., "Voice over 
IP via Bluetooth/Wi-Fi Peer to Peer," Advances in Engineering, Science and 
Management (ICAESM), 2012 International Conference on , vol., no., pp.828,837, 
30-31 March 2012

%ref 10
\bibitem{}
Kbar, G.; Mansoor, W.; Naim, A., "Voice over IP Mobile Telephony Using WIFI 
P2P," Wireless and Mobile Communications (ICWMC), 2010 6th International 
Conference on , vol., no., pp.268,273, 20-25 Sept. 2010. doi: 
10.1109/ICWMC.2010.97
 
%ref 11
\bibitem{}
Wang, Gang and Zhang, Chunhong and Qiu, Xiaofeng and Zeng, Zhimin , “An 
efficient relay node selection scheme to improve the performance of P2P-based 
VoIP applications in Chinese internet,” Multimedia Tools and Applications 64(3), 
pp. 599-625, doi: 10.1007/s11042-011-0952-5

%ref 12
\bibitem{}
Saikat Guha, Neil Daswani, “An Experimental Study of the Skype Peer-to-Peer 
VoIP System,” Computing and Information Science Technical Reports, 16-Dec-2005.

%ref 13
\bibitem{}
Tobias Hoßfeld, Andreas Binzenhöfer, “Analysis of Skype VoIP traffic in 
UMTS: End-to-end QoS and QoE measurements,” Computer Networks, Volume 52, Issue 
3, 22 February 2008, Pages 650-666, ISSN 1389-1286, 
http://dx.doi.org/10.1016/j.comnet.2007.10.008.

%ref 14
\bibitem{}
Baset, S. A., and Schulzrinne, H. (2004). An analysis of the skype 
peer-to-peer internet telephony protocol. arXiv preprint cs/0412017.

%ref 15
\bibitem{}
Fobert, J., Navaratnam, S., Dagert, P. J., and McKinnon, S. J. (2005). U.S. 
Patent No. 6,853,713. Washington, DC: U.S. Patent and Trademark Office.

%ref 16
\bibitem{}
Varga, A. (2001, June). The OMNeT++ discrete event simulation system. 
InProceedings of the European simulation multiconference (ESM’2001) (Vol. 9, No. 
S 185, p. 65). sn.

%ref 17
\bibitem{}
Anggadjaja, E., and McLoughlin, I. (2010, December). Point-to-point OMNeT++ 
based simulation of reliable transmission using realistic segmentation and 
reassembly with error control. In Advances in Computing, Control and 
Telecommunication Technologies (ACT), 2010 Second International Conference on 
(pp. 125-128). IEEE.

%ref 18
\bibitem{}
Bohge, M., and Renwanz, M. (2008, March). A realistic VoIP traffic generation 
and evaluation tool for OMNeT++. In OMNeT++ 2008: Proceedings of the 1st 
International Workshop on OMNeT++(hosted by SIMUTools 2008).

%ref 19
\bibitem{}
Qing, X., and Ren, X. (2012, June). Research of routing protocols simulation 
for wireless sensor networks based on OMNeT++. In Quality, Reliability, Risk, 
Maintenance, and Safety Engineering (ICQR2MSE), 2012 International Conference on 
(pp. 79-82). IEEE.

%ref 20
\bibitem{}
Postel, J., and Reynolds, J. K. (1983). Telnet protocol specification.

%ref 21
\bibitem{}
Fielding, R., Gettys, J., Mogul, J., Frystyk, H., Masinter, L., Leach, P., 
and Berners-Lee, T. (1999). Hypertext transfer protocol–HTTP/1.1.

%ref 22
\bibitem{}
Seckin, G., and Golshani, F. (2000). Real-time transmission of multilayer 
video over ATM networks. Computer Communications, 23(10), 962-974.


\end{thebibliography}

%\bibliography{pacrim15}


\end{document}
