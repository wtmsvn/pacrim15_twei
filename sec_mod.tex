\section{Network Scenario and Modeling}

\begin{figure}[h!]  
  \centering  
    \includegraphics[width=0.5\textwidth]{figures/model1_test2.jpg}
  \caption{P2P-based Model for VoIP}
  %\vspace{-1in}
  \label{fig:mod1}  
\end{figure}
There are four main kinds of components/nodes in the designed scenario, a 
presenter, multiple relayers and listeners, and a remote cloud server. Each 
component can coordinate with one another in order pass streaming content such 
as the voices or camera videos of the speaker, to the listeners in real-time. As 
showed in Fig.~\ref{fig:mod1}, the network is organized in a hierarchical 
architecture. The presenter will be producing the streaming content from the 
root node and broadcast to the relayers and listeners, as well as to 
the cloud servers for backup. 

For example, one can think of the model as a lecture in real life. A presenter 
is an instructor who gives a lecture at the front of the room, and the 
listeners are 
students who are listening to the lecture (can also be remote listening). When 
a 
presenter creates a presentation, the voices or videos being recorded will be 
sent to both the child nodes and the remote cloud server so that the servers can 
maintain a full copy for the streaming content in the presentation. All the 
other nodes can passively receive the packets from the presenter.
%The relayers and cloud server are possible approaches the instructor adopts to 
%pass on that voice stream. 
%How well a student understand is not a good 
%predictor of instructor’s capability; it largely depends on the media, that 
%is, how well the instructor could transfer his thinking in a explanatory way. 
%And that is where the cloud and relayers come into play.
There are different routes a voice packet can travel through. The 
listeners receive voice packets through the relayers while the cloud server 
can provide back-up services. If any of the listeners or relayers encounter a 
data loss due to a bad network condition, the node can send requests to the 
cloud server and retrieve the missing content. Once a relayer goes down, 
the listeners will request the voice stream from cloud. The cloud will take 
over 
and serve as the main approach to deliver packets. 
%Noted that only in simple cases where all attendees share the same role do 
%relayers act as routers. 
For a more complex scenario, where there are multiple roles for the listeners, 
relayers act as filter and only forward packets to the designated attendees. 

\begin{figure}[h!]  
  \centering
      \includegraphics[width=0.5\textwidth]{figures/model2_test2.jpg}
  \caption{Relayers Become Presenters}
  \label{fig:mod2}
\end{figure}

However, this simple hierarchical architecture cannot be adaptable to 
dynamic network changes and could experience an unbalanced network 
traffic problem which could seriously reduce the quality of serve and 
experience. For example, Fig.~\ref{fig:mod2} shows that if a 
high-level (near to the root) relayer suddenly leaves the network, all the 
nodes lower than the relayer will not be able to receive the streaming content 
any more, and thus the server will receive a large quality of requests from the 
nodes which loss the connection from the presenter. This will cause the server 
bandwidth to be occupied by these nodes which will make the network flow 
unbalanced. Also, the relayers among these nodes could lose their advantage in 
passing the content to the listeners, as the listeners could already obtain the 
requested packets from the cloud server and bandwidth between the relayers 
and the listeners is wasted. 

Therefore, our proposed solution is to design a P2P-based 
architecture among the relayers 
and listeners. The server could act as the tracker in this scenario. Every node 
will have to register with the server first when it joins in a presentation so 
that the server will be aware of the network situation for each node. According 
to the information the server received from each node (including bandwith, 
account information based on the future business model, etc.), some nodes will 
be selected as relayers at different levels, while the other nodes will act as 
the listeners (leaves) and establish connections with the relayers. The nodes 
at the same level will be connected as well, to establish the P2P-based 
architecture, as shown in Fig.~\ref{fig:mod3}.


\begin{figure}[h!]  
  \centering
      \includegraphics[width=0.5\textwidth]{figures/model3_test2.jpg}
  \caption{Relayers Become Presenters}
  \label{fig:mod3}
\end{figure}


\begin{comment}
Besides, the concept of background channel will be applied for transmitting the 
data in the background. Apart from the live channel which ensures the client to 
have the presentation on live, there will also be a background channel to help 
to upload or download previous missing data that make the data of the 
presentation synchronized and completed in both server and client side while the 
clients won’t even notice.
\end{comment}







