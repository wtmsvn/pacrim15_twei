\section{System Modeling}

Simulation \& Omnet++:
Omnet++ is used in the project as a simulation tool to explore the most suitable model and algorithm for ThinkTogether. After investigating potential networking models from above papers, the team will build models in Omnet++, where result data will be collected for analysis. The study done by Varga[16] provides brief introduction of Omnet++, including Model Libraries, NED language, model structure. The paper also describes how to execute the simulation under a powerful graphical user interface, which will help the team to better utilize the tool. Paper done by Anggadjaja, E[17]. features the use of Omnet++ based simulation for reliable point-to-point wireless transmission. The outcome could provide inspiration for employing Peer-to-peer model in wireless network. There are several other papers[18][19] explaining the main infrastructure and internal design of Omnet++. The team can gain overall understanding of the tool, through extensive papers above, and use Omnet++ with efficiency and accuracy. 

Experimental data is the most scientific and objective way of verifying the efficiency of a particular network. There comes the simulation. In the project, two network models will be built in order to observe the behavior of data packets transferred in between. One of networking protocol is Telnet, on which a paper done by J. Postel et al[20] provides specification and standard. This is a relatively old paper that is published in early 1980s. The paper draws a whole picture of telnet protocol from general consideration to signal processing. The team refers to this paper to get a understanding of how telnet protocol is put to use. Another protocol adopted is HTTP. There are wider range of papers dedicated to HTTP. One done by R Fielding[21] provides a full-scaled overview, which is used as a reference for modelling. 

Multilayer Transmission

ThinkTogether is a communication application that contains multiple layers in packet transmission. Data sent by each individual is handled differently.  A paper done by G. Sekin et al[22] provides an insight of how to manage multi priority traffic in ATM network. The study object is content-based video objects. As the common characteristics shared between video stream and voice stream, such as low tolerance in delay, jitter and packet misorder, our team could adopt ideas proposed in the paper. Unfortunately, there is no evaluation involved in the paper, therefore the feasibility of proposed algorithm remains unknown. 

