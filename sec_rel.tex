\section{Related Work}

\begin{comment}
Voice over IP (VoIP):

There are various studies that have been done in the field of VoIP (Voice 
over IP) which is a technology used 
for voice communication over multimedia devices with the help of the IP 
protocols. Some of the related work has been done by Wook \& Kang [1] where 
their main focus is based on VoIP using SIP( Session Initiation Protocol) for 
smartphones working on iOS, Android \& Windows based OS , In our project we’ll 
be using a similar technology of VoIP but our focus is on iOS devices i.e. 
mostly IPAD’s. Similarly there is some other research done by Fathi et al. [2] 
for the optimization of VoIP services in Wireless networks and how to reduce the 
disruptions in real-time sessions by focusing on the network layer and not so 
much on the link layer, In our project we aim to use a similar concept where we 
aim to reduce delay in voice streaming for live presentations for Wireless 
devices with a focus on both the network and link layer. Work done by Rao et al. 
[3] is based on providing VoIP services for GSM based devices without making any 
modifications in the GSM network, we use a similar concept for providing VoIP 
services but with a focus on iOS devices instead of GSM devices . 
There is some work done to improve the VoIP capacity in Wireless networks done 
by Fong et al. [4], we plan to use a similar concept as presented in the paper 
[4] to handle voice traffic as with the voice streams during a live presentation 
there is a chance that the voice traffic might increase and it is very important 
to handle this traffic in the best possible way to avoid any disruption in voice 
streaming in real-time. All these studies talk about VoIP applications and how 
to improve the capacity of such systems in Wireless networks.
Now there some research done for VoIP in mobile devices like: Aggregation of 
VoIP streams done by Komolafe \& Gardner [5], we focus on using a similar 
concept but instead of aggregation of calls we are concerned with aggregation of 
voice streams.
The work done by Agarwal et al. [6] is based on energy management of VoIP over 
Wi-Fi for Smartphones, In our project instead of focusing on Smartphones we 
focus on iOS devices (IPAD’s) , but we also use a similar concept for energy 
management of VoIP over the devices and with increasing amount voice streaming 
over the network the energy of the device needs to be managed in an efficient 
and environment friendly manner.  Quality Measurement of VoIP over mobile 
devices by Chen et al. [7] gives some methods to measure the Quality of VoIP 
over mobile devices, we plan to use a similar concept as voice quality is one 
the most important aspects in our project, so we need to ensure that voice 
quality is perfect without any kind of noise or disruption in voice streaming 
over the network. Performance Measurement of VoIP over the mobile networks by 
Kim et al. [8] provide some methods to measure the performance of VoIP, we will 
be using a similar approach as proposed in the paper [8] to measure the 
performance of VoIP over the mobile networks and by using the results of 
performance measurement we would analyse the network and try to improve any 
possible disruption that might cause any kind of delay in streaming voices over 
the network.
There are some other models where VoIP has been implemented using P2P 
(Peer-to-peer) like: Improving performance of P2P based VoIP applications by 
Wang et al. [11] where they analysed the delay in the Chinese internet system. 
To analyse the delay different methods were used in the paper [11], we aim to 
use a similar kind of technique to analyse the delay of voice streaming in our 
application.


Skype:
There are some existing technologies that use the VoIP feature like Skype. Lot 
of research has been done in this field like, An experimental study on Skype 
P2P VoIP Systems done by Guha \& Daswani [12] where they study different 
characteristics of network traffic from data derived from online users in the 
world, Based on the results obtained in this study we aim to analyse those 
results and will try to employ a similar approach used in the P2P VoIP 
technology used in Skype to our own project. Work  done by Hoßfeld \& 
Binzenhöfer [13] where they analyze the Quality of service (QoS) of the 
end-to-end connection and the Quality of Experience (QoE) of the end user, we 
aim to use a similar kind of approach to analyze the QoS \& the QoE of the 
application ThinkTogether so that if there is any problem in any of these 
measurements it can be fixed in the best possible way so that we can achieve our 
goal of implementing voice streaming in the application with the least amount of 
disturbance while transferring voice over the network. Analysis of the Skype P2P 
Internet telephony protocol by Baset and Schulzrinne [14] aims at studying the 
voice quality that has been implemented in Skype and how it is better than other 
applications that have also implemented a similar feature, In our project we aim 
to use a similar concept that has been implemented in Skype but our focus is 
only on the voice part and not the video part that has been implemented in 
Skype.
In our project we'll be using all these technologies and studying some other 
works that has been  done on them and use those ideas to implement VoIP function 
in ThinkTogether.


Peer-to-peer(P2P) \& Client/Server model(C/S):

There are two widely adopted networking model, Peer-to-peer(P2P) and 
Client-server(CS). Peer-to-peer is a networking mechanism in which each machine 
acts as server and clients at the same time. The roles are interchangeable 
simultaneously, dividing the tasks into a set of individual machines. In 
contrast to Server-Client model in which the supply of resource is separated 
from the consumption of resource, there is no explicit division in terms of work 
allocation in peer-to-peer networking. This eliminates the need to establish 
central hosts or servers for coordination. 

The concept of Peer-to-peer model used in transferring voice stream from one 
node to another has been discussed throughout years. Several connection models 
using P2P protocol have already been proposed to facilitate the communication 
between multiple handheld devices. A paper[9] done by S.Sundar et al put forward 
a novel algorithm along with suggested architecture to support voice 
transmission among mobile phones. Short-distance mobile communication is set up 
through Bluetooth, which is an integrated facility for most mobile phones 
nowadays; while long-distance calls are connected through WIFI. The mechanism 
developed in the paper proceed with the calls using GSM technology if no 
cost-free networking is accessible. Our primary focus is managing data while 
connected to network; yet the knowledge using GSM provides insight of future 
development. Ghassan Kbar et al[10] also dive in  research of similar topic. 
These two papers concentrate on eliminating costly GSM(Global system for Mobile) 
telecommunication by exploring other channels such as Bluetooth and WIFI. The 
work presented in papers relates voice implementation in ThinkTogether app, 
typically in the first stage - connection. ThinkTogether project absorbs the 
idea of how to design communication model to be adaptive of current network 
situation. 

To compare with Peer-to-peer, another networking mechanism:server-client model 
is also introduced in the project. Each model will be evaluated in terms of 
performance in voice data transmission. The one with better performance in voice 
stream will be chosen. The paper proposed by Fobert et al[15] provides a general 
look into IP telephony in Server/Client Model. It features components included 
in managing voice data packets. The paper was published in early years when IP 
telephony was immature, therefore only providing a fundamental principle of 
transferring voice data over internet in C/S model. However, simulation will 
built on that knowledge to determine which candidate model, either C/S or P2P, 
benefits ThinkTogether the most.

\end{comment}


Our case study focuses specifically on VoIP (Voice over IP), which is a 
technology used for voice communication over multimedia devices with the help of 
the IP protocols. Wook \& Kang \cite{HS_VoIP_ICTC10} focused on VoIP using SIP 
(Session 
Initiation 
Protocol) for smartphones in general, whereas we hope to extend some of these 
results to iPads.  Fathi et al.~\cite{HSR_VoIP_TVT07} optimized VoIP services 
in wireless networks 
and reduced the disruptions in real-time by focusing on the network layer as 
opposed to the link layer.   In our work we aim to use a similar concept, where 
we aim to reduce delay in voice streaming for live presentations for wireless 
devices with a focus on both the network and link layer.  
 
There are many studies on VoIP applications and how to improve the capacity of 
such systems in Wireless networks.  For example, Rao et 
al.~\cite{HYS_VoIP_COM00} provided VoIP 
services for GSM (Global System for Mobile) based devices without making any 
modifications in the GSM network.  Similar work by Fong et 
al.~\cite{MRSR_VoIP_COM08} focused on 
Windows.  Approaches like these are great examples of how to handle traffic 
dynamically, to avoid any disruption in voice streaming in real-time. Komolafe 
\& Gardner~\cite{OR_VoIP_PMC03} studied aggregation of VoIP calls, whereas we 
focus on 
aggregation 
of voice streams. Agarwal et al. \cite{YRAP_VoIP_MSYS07} consider energy 
management of VoIP over 
Wi-Fi for Smartphones, which should be directly applicable to our case study 
using iPads.  An increase in voice streaming will most definitely require the 
energy of the device to be managed in an efficient and environmentally friendly 
manner.   Chen et al. \cite{WPY_VoIP_SYS11} provide methodology to measure the 
Quality of VoIP 
over mobile devices, and Kim et al. \cite{DHMS_WiMAX_WOW08} provide metrics for 
performance of VoIP 
over the mobile networks, and we intend to leverage these same methodologies in 
our work.  Related work studying VoIP using P2P design includes Wang et al. 
\cite{GCXZ_P2PVoIP_MTA13}, where they analyzed the delay in the Chinese internet 
system, which also 
factors in as an important concern in our case study.
 
An experimental study on Skype P2P VoIP Systems by Guha \& Daswani 
\cite{SN_Skype_CIST05} showed 
the variety of characteristics of network traffic from data derived from online 
users in the world.  We hope to leverage these insights in our work, along with 
work by  Hoßfeld \& Binzenh ofer \cite{TA_Skype_COMPNET08} focused on Quality of 
service (QoS) of 
the 
endto-end connection and the Quality of Experience (QoE) of the end user. A 
similar analysis of the Skype P2P Internet telephony protocol by Baset and 
Schulzrinne \cite{SH_Skype_INFO06} establishes the superior voice quality in 
Skype relative to 
other state-of-the-art applications.
 
Sundar et al \cite{SMPM_VoIP_ICAESM12} put forward a novel algorithm along with 
suggested architecture 
to support P2P voice transmission among mobile phones.  Short-distance mobile 
communication is set up through Bluetooth, while long-distance calls are 
connected through WIFI. The mechanism developed uses GSM technology if no 
costfree networking is accessible. Our primary focus is managing data while 
connected to network; yet GSM provides insight of future development.  Ghassan 
Kbar et al \cite{GWA_P2P_VoIP_ICWMC10} also concentrate on eliminating costly 
GSM telecommunication by 
exploring other channels such as Bluetooth and WIFI.  Our goal is to extend the 
voice feature of the ThinkTogether app to be adaptive of current network 
situation.
 
Fobert et al \cite{JSPS_USPatent05} provides a general look into IP telephony 
in Server/Client 
Model managing voice data packets.  The paper was published in early years when 
IP telephony was immature, therefore only providing a fundamental principle of 
transferring voice data over internet  We use this as a comparison to the 
proposed P2P architecture in terms of performance in voice data transmission 
within the ThinkTogether app.




