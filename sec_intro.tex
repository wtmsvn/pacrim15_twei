\section{Introduction}

\begin{comment}
ThinkTogether is a mobile collaboration platform developed by Two Tall Totems, which is also a creative Bring Your Own Device (BYOD) solution. It enables teachers and students to cooperate together through the lecture in a class or presenters and attendees to participate together in a conference. The current implementation of the project on iOS supports the most functionality of the presentation, yet not with voice stream. It is essential to support the voice stream in the application for remote education since students will be able to have a better interaction with the class even if they are not in the classroom. We are going to design and implement a Peer-to-Peer (P2P) based Voice over IP (VoIP) system on the top of the existing application on iOS so that the users not only could actively enjoy the presentation on live but also could store and replay it later on the cloud.

Due to a variety of factors that could affect the quality of service (QoS) for the application, such as limited bandwidth, high volume traffic, network delay and effect of peer signal, we are going to explore and analyze a better solution with optimized performance. This project aims to create a new education model and subvert the traditional way of learning, sharing and communicating with mobile devices through different cutting-edge mobile technology. First of all, we will model a hierarchical architecture, and verify the model as well as the design with a simulation software Omnet++. After that, analysis will be applied to the data collected from the results of the simulation and we are going to discover the bottleneck in our VoIP solution as well as make an improvement.

Therefore, we propose to experiment the method of pushing data among all the 
devices and ensure the usage of bandwidth that is enough for each user as well 
as for the cloud server. We are going to implement a stable service based on 
Voice over IP (VoIP), Peer-to-peer (P2P) or Multi-peer technology, which would 
also be scalable for large amount users with mobile devices in the existing App. 
With our partner Two Tall Totems, development on this project with these 
advanced communication and network technology could greatly change how people 
think together and help to build a better world.
\end{comment}

%background
Collaboration platforms for Computer Supported Collaborative Work (CSCW) enable 
users to leverage technology to cooperatively develop shared content.   
Successful platforms now support everything from education to medicine, 
typically supported through a centralized service or cloud. Though read-only 
content distribution has been successfully shared between 
peers~\cite{XG_P2P_INFO04}, mutable content with complex interactions and 
dependencies 
has been more difficult to decentralize. Essentially, the problem is that 
decentralized approaches require coordination as nodes can come and go, and 
connections can fail, creating a myriad of different failure modes.

%problem
In this paper, we overview the design of a Peer-to-Peer (P2P) infrastructure for 
a CSCW application, ThinkTogether.  ThinkTogether is an iOS prototype for 
creating and sharing content in almost real-time. We consider the specific 
problem of Voice over IP (VoIP), as it is representative of a stream of content 
that is particularly sensitive to service disruptions.  Our goal is to be able 
to share all content, including voice, live, but also allow users to replay it 
later based on access to a shared server or the cloud.   Due to the variety of 
factors that could affect the quality of service (QoS) for the application, such 
as limited bandwidth, high volume traffic, network delay and the effects of 
other peers, we propose a dynamically decentralized approach, in order to 
optimized performance as conditions in the system change.  We considering this 
in a case study for an educational platform, which could ultimately subvert the 
traditional way of learning, sharing and communicating with mobile technology.
 

%section description
This paper is organized as follows.  After an overview of related work (Section 
II), we present our model of a hierarchical architecture, and provide the 
verification of the model (Section III) as well as the design with simulation 
software Omnet++~\cite{A_OMNET_ESM01}. We then describe the ways in which 
the simulation results inform the implementation of the system.  
In particular, we identify bottlenecks to anticipate in the VoIP educational 
platform case study. The analysis underscores the importance to augment the 
existing prototype with a means of easily distributing data throughout the peers 
in the system, in a way that provides adequate bandwidth for both the current 
users and the cloud-based content capture (Section IV).  We generalize this 
construct, and propose that if we treat the platform as a {\em distributed 
cloud}, it may be able to address the key challenge---to scale across 
potentially thousands of mobile devices, all creating and modifying shared 
content, in a globally distributed system (Section V).  We conclude with an 
outline of future work (Section VI).

\begin{comment}
This paper is organized as follows.  We first present our model of a 
hierarchical architecture, and provide the verification of the model as well as 
the design with simulation software Omnet++~\cite{CITATION} (Section 1). We then 
describe the ways in which the simulation results inform the implementation of 
the system (Section 2).  In particular, we identify bottlenecks to anticipate in 
the VoIP educational platform case study.  The analysis underscores the 
importance to augment the existing prototype with a means of easily distributing 
data throughout the peers in the system, in a way that provides adequate 
bandwidth for both the current users and the cloud-based content capture 
(Section 3).  We generalize this construct, and propose that if we treat the 
platform as a {\em distributed cloud}, it may be able to address the key 
challenge---to scale across potentially thousands of mobile devices, all 
creating and modifying shared content, in a globally distributed system (Section 
4).  We conclude with an outline of future work (Section 5).
\end{comment}